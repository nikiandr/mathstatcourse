\documentclass{article}
\usepackage[utf8]{inputenc}
\usepackage[T1]{fontenc}
\usepackage[ukrainian]{babel}
\usepackage[12pt]{extsizes}
\usepackage{graphicx}
\usepackage{amsmath}
\usepackage{amsfonts}
\usepackage{multicol}
\graphicspath{{pictures/}}
\DeclareGraphicsExtensions{.pdf,.png,.jpg}

\title{Теорвер}
\author{nikita.forduy }
\date{February 2020}

\usepackage{natbib}
\usepackage{graphicx}

\begin{document}
\pagestyle{empty}

\begin{titlepage}
    \thispagestyle{empty}
    \setlength{\parindent}{0ex} % set paragraph indenting to zero
    
    \begin{center}
      НАВЧАЛЬНО-НАУКОВИЙ КОМПЛЕКС \\
      "ІНСТИТУТ ПРИКЛАДНОГО СИСТЕМНОГО АНАЛІЗУ" \\
      НАЦІОНАЛЬНОГО ТЕХНІЧНОГО УНІВЕРСИТЕТУ УКРАЇНИ \\
      "КИЇВСЬКИЙ ПОЛІТЕХНІЧНИЙ ІНСТИТУТ ІМЕНІ ІГОРЯ СІКОРСЬКОГО" \\
      \smallskip
      КАФЕДРА МАТЕМАТИЧНИХ МЕТОДІВ СИСТЕМНОГО АНАЛІЗУ \\
    \end{center}
    \vspace{60mm}
    
    \begin{center}
      РОЗРАХУНКОВА РОБОТА \\
      з предмету "Математична статистика" \\
    \end{center}
    
    \vspace{30mm}
    

    \hfill
    \begin{minipage}{.4\linewidth}
      \begin{flushright}
        Виконав студент групи КА-81
        Фордуй Нікіта
        \smallskip
        Перевірила Каніовська І.Ю.
      \end{flushright}
    \end{minipage}
    
    \vspace{10mm}

    \vfill
    \begin{center}
      Київ 2020
    \end{center}
    
    \setlength{\parindent}{5ex} % reset paragraph indenting
\end{titlepage}

\pagestyle{plain}

\large
\section{Завдання}
Дана конкретна реалізація вибірки об’ємом n = 100:
\newline
\newline
\begin{tabular}{cccccccccccccccccccc}
  \ttfamily 2 & 0 & 8 & 0 & 15 & 1 & 1 & 1 & 7 & 1 & 0 & 0 & 
  3 & 1 & 1 & 1 & 0 & 0 & 3 & 1 \\
  \ttfamily 2 & 4 & 10 & 6 & 1 & 0 & 1 & 0 & 0 & 2 & 0 & 1 & 
  5 & 0 & 1 & 9 & 4 & 2 & 11 & 3\\
  \ttfamily 2 & 0 & 8 & 1 & 6 & 3 & 0 & 1 & 1 & 4 & 0 & 9 & 
  5 & 3 & 3 & 0 & 0 & 10 & 2 & 0\\
  \ttfamily 3 & 11 & 0 & 9 & 0 & 1 & 4 & 1 & 0 & 2 & 0 & 1 & 
  1 & 3 & 4 & 7 & 1 & 3 & 3 & 0 \\
  \ttfamily 4 & 7 & 6 & 0 & 3 & 0 & 1 & 15 & 11 & 1 & 2 & 4 & 
  0 & 2 & 0 & 0 & 0 & 26 & 4 & 0
\end{tabular}
\newline
\section{Побудова варіаційного ряду вибірки }
Маємо невелику кількість різних значень - тому побудуємо 
дискретний варіаційний ряд.
Підрахувавши кількість варіант (14) та їх частоти і знаючи 
об’єм вибірки отримаємо 
дискретний варіаційний ряд :
\newline
\newline
\begin{tabular}{|p{44pt}|p{13pt}|p{13pt}|p{13pt}|p{13pt}
  |p{13pt}|p{13pt}|p{13pt}|p{13pt}|p{13pt}|p{13pt}|p{13pt}
  |p{13pt}|p{13pt}|p{13pt}|}
  \hline
  $x_i^*$& 0 & 1 & 2 & 3 & 4 & 5 & 6 & 7 & 8 & 9 & 10 & 11 & 15 & 26\\
  \hline
  $n_i$& 29 & 22 & 9 & 11 & 8 & 2 & 3 & 3 & 2 & 3 & 2 & 3 & 2 & 1\\
  \hline
  $\omega_i = \frac{n_i}{n}$ & $\frac{29}{100}$ & 
  $\frac{22}{100}$ & $\frac{9}{100}$ & $\frac{11}{100}$ & 
  $\frac{8}{100}$ & $\frac{2}{100}$ & $\frac{3}{100}$ 
  & $\frac{3}{100}$ & $\frac{2}{100}$ & $\frac{3}{100}$ & 
  $\frac{2}{100}$ & $\frac{3}{100}$ & $\frac{2}{100}$ & 
  $\frac{1}{100}$\\
  \hline
  $\omega_i^H$&$\frac{29}{100}$&$\frac{51}{100}$&$\frac{60}{100}$&
  $\frac{71}{100}$&$\frac{79}{100}$&$\frac{81}{100}$&$\frac{84}{100}$&
  $\frac{87}{100}$&$\frac{89}{100}$&$\frac{92}{100}$&$\frac{94}{100}$&
  $\frac{97}{100}$&$\frac{99}{100}$&1\\
  \hline
\end{tabular}
\newline
\newline
де $x_i^*$ - варіанти реалізації вибірки, $n_i$ - частота 
варіанти, $\omega_i = \frac{n_i}{n}$ - частість варіанти або 
відносна частота, $\omega_i^H$ - накопичена частість.
\newpage
За дискретним варіаційним рядом побудуємо його геометричну 
інтерпретацію - полігон відносних частот (частостей):
\newline
\includegraphics[scale = 0.9]{pol}
Порівняємо полігон частостей нашої реалізації 
виборки із полігоном ймовірностей закону Паскаля при 
різних значеннях його параметра.
\newline
\includegraphics[scale = 0.8]{pol+pas1}
\newline
\includegraphics[scale = 0.8]{pol+pas2}
\newline
\includegraphics[scale = 0.8]{pol+pas3}
\newline
Можна побачити, що полігон ймовірностей закону Паскаля
при певних значеннях його параметра (a = 1, 2, 3, 4) 
дуже схожий на полігон частостей нашої реалізації вибірки.
\newpage
\section{Емпірична функція розподілу}
Побудуємо емпіричну функцію розподілу за вже побудованим 
дискретним варіаційним рядом:
\newline
\begin{equation}
  F_n^*(x) = \begin{cases}
    0,  & x \leq 0 \\
    \frac{29}{100}, & 0 < x \leq 1 \\
    \frac{29 + 22}{100} = \frac{51}{100}, & 1 < x \leq 2 \\
    \frac{51}{100} + \frac{9}{100} = \frac{60}{100}, & 2 < x \leq 3 \\
    \frac{60}{100} + \frac{11}{100} = \frac{71}{100}, & 3 < x \leq 4 \\
    \frac{71}{100} + \frac{8}{100} = \frac{79}{100}, & 4 < x \leq 5 \\
    \frac{79}{100} + \frac{2}{100} = \frac{81}{100}, & 5 < x \leq 6 \\
    \frac{81}{100} + \frac{3}{100} = \frac{84}{100}, & 6 < x \leq 7 \\
    \frac{84}{100} + \frac{3}{100} = \frac{87}{100}, & 7 < x \leq 8 \\
    \frac{87}{100} + \frac{2}{100} = \frac{89}{100}, & 8 < x \leq 9 \\
    \frac{89}{100} + \frac{3}{100} = \frac{92}{100}, & 9 < x \leq 10 \\
    \frac{92}{100} + \frac{2}{100} = \frac{94}{100}, & 10 < x \leq 11 \\
    \frac{94}{100} + \frac{3}{100} = \frac{97}{100}, & 11 < x \leq 15 \\
    \frac{97}{100} + \frac{2}{100} = \frac{99}{100}, & 15 < x \leq 26 \\
    \frac{99}{100} + \frac{1}{100} = 1, & x > 26 \\
  \end{cases}
\end{equation}
\newpage
Зобразимо емпіричну функцію розподілу геометрично :
\newline
\includegraphics[scale = 0.8]{func}
\newline
Порівняємо графік емпіричної функції розподілу варіаційного ряду 
з графіком функції розподілу закону Паскаля при 
різних параметрах(a = 1, 2, 3):
\newline
\includegraphics[scale = 0.8]{func+geom4}
\newline
\includegraphics[scale = 0.8]{func+geom3}
\newline
\includegraphics[scale = 0.8]{func+geom2}
\newline
З рисунків вище можна побачити що графік емпіричної функції
розподілу нашої реалізації вибірки схожий при певних значеннях
параметра a на функцію розподілу закону Паскаля.
\section{Обчислення вибіркових характеристик генеральної 
сукупності (медіана, мода, ассиметрія)}
Для початку знайдемо $({Mo}_\xi^*)_{\text{знач.}}$ - 
значення вибіркової моди (тієї варіанти, якій відповідає 
найбільша частість). Для знаходження цієї варіанти використаємо
вже побудований дискретний варіаційний ряд (див. ст. 1 пункт 2).
Проаналізувавши варіаційний ряд побачимо, що:
$$({Mo}_\xi^*)_{\text{знач.}} = x_1^* = 0$$
Зауважимо, що випадкова величина $\mu$, розподілена за 
законом Паскаля при будь-яких значеннях параметра a
має моду ${Mo}_\mu = 0$.
\newline
\newline
Знайдемо значення вибіркової медіани $({Me}_\xi^*)_{\text{знач.}}$ 
для нашої реалізації виборки. З варіаційного ряду (див. ст. 1 
пункт 2), враховуючи те, що кількість варіант - парна, 
знайдемо:$$ ({Me}_\xi^*)_{\text{знач.}} = \frac{x_7^* + x_8^*}
{2} = 6.5 $$
\newline
Для знаходження значення вибіркової ассиметрія спочатку потрібно 
знайти значення вибіркової дисперсії, а тому й вибіркового середнього: 
$$\overline{x} = (E^*_{\xi})_{\text{знач.}} = \frac{1}{100} 
\sum_{k = 1}^{14} x_k^* n_k = 3.06$$
За допомогою цього знайдемо значення вибіркової дисперсії:
$$(D^*_{\xi})_{\text{знач.}} = \frac{1}{100} \sum_{k = 1}^{100}
(x_k - 3.06)^2 = 17.136400000000002$$
Отримавши значення вибіркової дисперсії, можна отримати значення
вибіркової ассиметрії для даної реалізації вибірки:
$$({As}_{\xi}^*)_{\text{знач.}} = \frac{\frac{1}{100}
\sum_{k = 1}^{100}(x_k - 3.06)^3}{(17.136400000000002)^
{\frac{3}{2}}} = 2.504088773053977$$
\section{Незміщені оцінки математичного сподівання та дисперсії}
$\xi$ - генеральна сукупність, $\vec{\xi} = 
(\xi_1, \xi_2, \dots, \xi_{n})$ - випадкова вибірка, 
n = 100 - об’єм вибірки.  
\newline
За точкову оцінку математичного сподівання візьмемо вибіркове середнє:
$${E_\xi}^* = \frac{1}{n} \sum_{k=1}^{n}\xi_k$$
Перевіримо незміщенність цієї точкової оцінки:
\begin{equation}\label{bias}
  E({E_\xi}^*) = E(\frac{1}{n} \sum_{k=1}^{n}\xi_k) = 
  \frac{1}{n} \sum_{k=1}^{n}E_{\xi_k} = \frac{1}{n}nE_{\xi} 
  = E_\xi
\end{equation}
Відповідно, ця точкова оцінка матсподівання є незміщеною.
\newline
За точкову оцінку дисперсії візьмемо:
$$ {D_{\xi}}^{*} = \frac{1}{n} \sum_{k=1}^n 
(\xi_k - \overline{\xi})^2 = \frac{1}{n}\sum_{k=1}^n
((\xi_k - E_\xi) - (\overline{\xi} - E_\xi))^2 = $$
$$= \frac{1}{n} \sum_{k = 1}^n(\xi_k - E_\xi)^2 - 2(\overline{\xi} 
- E_\xi)\frac{1}{n}\sum_{k=1}^n(\xi_k - E_\xi) + (\overline{\xi} 
- E_\xi)^2 = $$
$$= \frac{1}{n} \sum_{k = 1}^n (\xi_k - E_\xi)^2 - (\overline{\xi} 
- E_\xi)^2$$
Порахуємо матсподівання цієї оцінки:
$$E({D_{\xi}}^{*}) = \frac{1}{n}\sum_{k=1}^n E(\xi_k - E_\xi)^2 
- E(\overline{\xi} - E_\xi)^2 = D_\xi - D_{\overline{\xi}}$$
Бачимо, що ця оцінка - зміщена.
Знайдемо $D_{\overline{\xi}} = \frac{1}{n}\sum_{k=1}^n D_{\xi_k} = 
\frac{D_{\xi}}{n}$.
$$E(D_\xi^*) = D_\xi - \frac{D_\xi}{n} = \frac{n-1}{n}D_\xi$$
Тоді оцінка $D^{**}_\xi = \frac{n}{n-1}D_\xi^*$ буде незміщеною 
оцінкою дисперсії.
$$D^{**}_\xi = \frac{n}{n-1} \frac{1}{n} \sum_{k=1}^n 
(\xi_k - \overline{\xi})^2 = \frac{1}{n - 1} \sum_{k=1}^n 
(\xi_k - \overline{\xi})^2$$
Обчислимо значення цих точкових оцінок на данній реалізації виборки:
$$(E^*_{\xi})_{\text{знач.}} = \frac{1}{100} 
\sum_{k = 1}^{14} x_k^* n_k = 3.06$$
де $x_k^*$ - к-та варіанта, $n_k$ - частота вибірки.
$$(D^{**}_\xi)_\text{знач.} = \frac{1}{99} \sum_{k = 1}^{100}
(x_k - 3.06)^2 = 17.30949494949495$$
\newpage
\section{Гіпотеза про розподіл, за яким отримано вибірку}
Виходячи з того що:
\begin{itemize}
  \item полігон частостей реалізації вибірки схожий на полігон 
  ймовірностей закону Паскаля (див. с. 3-4)
  \item емпірична функцію розподілу реалізації вибірки схожа на
  функцію розподілу закону Паскаля (див. с. 6-7)
  \item мода випадкової величини, розподіленої за законом Паскаля
  дорівнює 0 при будь-яких параметрах закону; в той же самий час 
  значення вибіркової медіани для нашої реалізації виборки також
  дорівнює 0.
  \item в наступному параграфі буде показано, що оцінка математичного 
  сподівання закона Паскаля ${E_\xi}^* = \frac{1}{n} 
  \sum_{k=1}^{n}\xi_k$ є не тільки 
  незміщенною, а й конзистентною та ефективною. Тоді, якщо порівняти 
  полігон частостей даної вибірки та полігон ймовірностей, графік 
  емпіричної функції розподілу та графік функції розподілу закона 
  Паскаля з відповідним параметром, то вони будуть дуже схожі(див. 
  наступні рис.)
\end{itemize}
\includegraphics[scale = 0.8]{hypotesis1.png}
\newline
\includegraphics[scale = 0.8]{hypotesis2.png}
\newline
Таким чином висувається гіпотеза, що генеральна сукупність, 
якою породжена данна вибірка, розподілена за законом Паскаля.
\newpage
\section{Точкові оцінки параметру гіпотетичного закону розподілу}
Спочатку скористаємось методом моментів для знаходження точкової 
оцінки параметра a закона Паскаля ( $\mu \sim Pas(a)$ ).
\newline
Прирівняємо емпіричний початковий момент 1-го порядку та математичне 
сподівання випадкової величини, розподіленої за законом Паскаля; 
отримаємо рівняння Пірсона: 
$$E_\mu = E^*_\mu$$
$$E_\mu = a, E^*_\mu = \frac{1}{n} \sum_{k = 1}^n \xi_k$$
\begin{equation}
  (a^*)_\text{мм} = \frac{1}{n} \sum_{k = 1}^n \xi_k
\end{equation}
Отримали статистику - точкову оцінку параметра a закону Паскаля.
\newline
Тепер отримаємо точкову оцінку параметра а за допомогою методу 
максимальної правдоподібності (Фішера). Спочатку знайдемо функцію 
правдоподібності закона Паскаля: 
$$\mathcal{L}( \vec{x}, a ) = \prod_{k = 1}^n \mathbb{P} 
\{\xi = x_k\} = \prod_{k = 1}^n \frac{a^{x_k}}{(1 + a)^{x_k + 1}} = 
\frac{a^{\sum_{k =1}^n x^k}}{(1 + a)^{\sum_{k =1}^n x^k + n}}$$
$$\ln \mathcal{L}( \vec{x}, a ) = (\sum_{k=1}^n x_k)\ln a - 
((\sum_{k=1}^n x_k) + n)\ln(1+a)$$
\begin{equation}
  \frac{\partial\ln \mathcal{L}( \vec{x}, a )}{\partial a} = 
  \frac{1}{a(1+a)}\sum_{k=1}^n x_k - \frac{n}{1+a} = 0
\end{equation}
$$\frac{1}{a(1+a)}\sum_{k=1}^n x_k = \frac{n}{1+a}$$
$$a = \frac{1}{n} \sum_{k=1}^n x_k$$
Отримали оцінку параметра а закона Паскаля методом 
максимальної правдоподібності:
\begin{equation}
  (a^*)_\text{мп} = \frac{1}{n}\sum_{k=1}^n \xi_k
\end{equation}
Перевіримо виконання достатньої умови:
$$\frac{\partial^2\ln \mathcal{L}( \vec{x}, a )}{\partial^2 a} = 
-\frac{1+2a}{a^2(1+a)^2}\sum_{k=1}^n x^k + \frac{n}{(1+a)^2}$$
\begin{equation}
  \left.{\frac{\partial^2\ln \mathcal{L}( \vec{x}, a )}{\partial^2 a}}
  \right|_{a = a^*} = -\frac{n^2}{(1+a^*)^2}({\frac{1}{a^*} + 1}) < 0
\end{equation}
Достатня умова виконана.
\newline
Обома методами отримали однакову оцінку параметра а:
$$(a^*)_{\text{мм}} = (a^*)_{\text{мп}} = a^* = \frac{1}{n}
\sum_{k=1}^n \xi_k = \overline{\xi}$$
Перевіримо властивості цієї оцінки:
\begin{enumerate}
  \item \textbf{Незміщенність.} Вже доведена раніше (див. 2 cтор.9)
  \item \textbf{Конзистентність.} Так як $\{\xi_k\}$ - i.i.d\footnote{
    i.i.d. - Independent and identically distributed random variables 
    -  незалежні та однаково розподілені випадкові величини
  }
  , для $\forall \xi_k :E_{\xi_k} = a < \infty, D_{\xi_k} = a^2 + a - 
  \text{рівномірно обмежені}$, то за ЗВЧ
  \newline 
  $a^*\xrightarrow[n\to\infty]{\mathbb{P} }E_{a^*} = a$.
  А це і означає що оцінка конзистентна.
  \newpage
  \item \textbf{Ефективність.} Розглянемо вираз 
  $\frac{\partial\ln \mathcal{L}( \vec{x}, a )}{\partial a}$.
  Його вже було знайдено раніше ( див. 4 стор. 12 ).
  $$\frac{\partial\ln \mathcal{L}( \vec{x}, a )}{\partial a} = 
  \frac{1}{a(1+a)}\sum_{k=1}^n x_k - \frac{n}{1+a} = $$
  $$= \frac{n}{a(1+a)}(\frac{1}{n}\sum_{k=1}^n x_k - a) = $$
  $$= C(n, a)(a^* - a)$$
  Таким чином, за наслідком з нерівності Рао-Крамера, оцінка $a^* = 
  \frac{1}{n}\sum_{k=1}^n \xi_k$ є ефективною.
\end{enumerate}
Таким чином, маємо незміщенну, конзистентну та еффективну оцінку 
параметра a закона Паскаля.
\section{Перевірка гіпотези про розподіл}
Перевірка гіпотези про розподіл генеральної сукупності буде здійснюватись 
за допомогою критерію $\chi^2$ (Пірсона)  з рівнем значущості $\alpha = 0.05$.
\newline
Висунемо гіпотезу $H_0: \xi \sim Pas(3.06)$. Згідно нашої гіпотези, 
генеральна сукупність може приймати такі значення: $\{0, 1, 2, \dots\}$.
Розіб'ємо цю множину на такі підмножини $X_i$, $i = \overline{0,5}$:

\begin{multicols}{2}
  \begin{itemize}
    \item $X_0 = \{0\}$
    \item $X_1 = \{1\}$
    \item $X_2 = \{2\}$
    \item $X_3 = \{3\}$
    \item $X_4 = \{4, 5\}$
    \item $X_5 = \{6, 7, \dots\}$
  \end{itemize}
\end{multicols}
\newpage
Обчислимо ймовірності $p_i = \mathbb{P}(\xi \in X_i/H_0)$ та 
$n_i$ - кількість значень реалізації вибірки, що потрапили в 
$X_i$.
\newline
\begin{tabular}{|l|l|l|l|l|l|l|}
  \hline
  $X_i$ & $\{0\}$ & $\{1\}$ & $\{2\}$ & $\{3\}$ & $\{4, 5\}$ & 
  $\{6, 7, \dots\}$ \\
  \hline
  $p_i$ & $0.2463$ & $0.1856$ & $0.1399$ & $0.1055$ & 
  $0.1394$ & $0.1833$\\
  \hline
  $n\cdot p_i$ & $24.63$ & $18.56$ & $13.99$ & $10.55$ & 
  $13.94$ & $18.33$\\
  \hline
  $n_i$ & 29 & 22 & 9 & 11 & 10 & 19 \\
  \hline
\end{tabular}
\newline
\newline
Бачимо, що $\sum_{i = 0}^{5}p_i = 1$, r = 6, і виконується умова 
\newline
$\forall i$ : 
$np_i \geq 10$.
Обчислимо значення статистики:
$$\eta = \sum_{i=1}^r\frac{(n_i - np_i)^2}{np_i}$$
$$\eta_\text{знач.} = \frac{(29 - 24.63)^2}{24.63} + 
\frac{(22 - 18.56)^2}{18.56} + \frac{(9 - 13.99)^2}{13.99} + $$
$$+ \frac{(11 - 10.55)^2}{10.55} + \frac{(10 - 13.94)^2}{13.94} + 
\frac{(19 - 18.33)^2}{18.33} \approx 4.35007$$
$r - s - 1 = 6 - 1 - 1 = 4, \alpha = 0.05$, тому за таблицею розподілу 
Пірсона знайдемо значення $t_{0.05, 4} = 9.5$. Бачимо, що 
$\mu_\text{знач.} < t_{0.05, 4}$. Робимо висновок, що на рівні 
значущості 0.05 дані не суперечать висунутій гіпотезі про те, що 
генеральна сукупність розподілена за законом Паскаля($\xi 
\sim Pas(3.06)$).
\end{document}
